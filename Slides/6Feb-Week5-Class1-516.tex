\documentclass{beamer}
\usepackage[utf8]{inputenc}
\usepackage{graphicx}

\author[Sowmya Vajjala]{Instructor: Sowmya Vajjala}


\title[ENGL 516X]{ENGL 516X: \\ Methods of Formal Linguistic Analysis}
\subtitle{Semester: Spring '18}

\date{6 Feb 2018}

\institute{Iowa State University, USA}

%%%%%%%%%%%%%%%%%%%%%%%%%%%

\begin{document}

\begin{frame}\titlepage
\end{frame}

\begin{frame}%2minutes
\frametitle{Class outline}
\begin{itemize}
\item Assignment 2 discussion
\item Strings in Python - overview
\item Formatting strings
\item Assignment 3 Description
\item Practice Exercises
\end{itemize}
\end{frame} %Chapter 5 in the text boo

\begin{frame}
\frametitle{}
\Large Assignment 2 discussion
\end{frame}

\begin{frame}[fragile]
\frametitle{Question 1}
\begin{itemize}
\item Note: my description was: "I will provide you a starter snippet, which asks for two numeric inputs from
user. Your task is to edit what I gave you, and make it work."
\end{itemize} \scriptsize
\begin{verbatim}
num1 = input("Enter a number: ")
num2 = input("Enter another number: ")
#Write your logic to test whether num1 and num2 are numbers below:
is_num1_numeric = num1.isnumeric()
is_num2_numeric = num2.isnumeric()
if is_num1_numeric and is_num2_numeric:
#write a line to print the sum of num1 and num2. 
      print(int(num1) + int(num2))
\end{verbatim}
\end{frame}

\begin{frame}[fragile]
\frametitle{Question 2}
(I asked to save this as a python file)
\begin{verbatim}
str = input("Enter a string: ")
if str.isalpha():
    print(str + "es")
else:
    print("please enter a proper string")
\end{verbatim}
\end{frame}

\begin{frame}[fragile]
\frametitle{Question 3}
(I asked to save this as a python file)
\begin{verbatim}
num = input("Enter a number: ")
if num.isnumeric():
   if int(num)%2 == 0:
         print("even")
   else:
         print("odd")
else:
   print("Enter a number!")
\end{verbatim}
\end{frame}

\begin{frame}
\frametitle{Question 4}
\begin{itemize}
\item put parantheses around 1-2
\item "PYTHON".isupper(), "PYTHON".lower()
\item 6**2 or 6*6; 6**3 or 6*6*6 (You can also use math module in Python, but I was looking for simpler solutions).
\item len("python")
\end{itemize}
note: These assignments are deliberately made easier than classwork (\textbf{something} should be easy to do!)
\end{frame}

%CODE solutions for Last class stuff
\begin{frame}[fragile]
\frametitle{Programs from last class}
\begin{itemize}
\item Program with multiple functions
\item Program to print multiplication tables
\end{itemize}
- see the posted solutions in the forum for Thursday. My solutions are also uploaded on Canvas.

\bigskip

If you go through and understand what happened, and write a program like that yourself without looking at the solution, you are good (for now!). 
\end{frame}

\begin{frame}[fragile]
\frametitle{Strings in Python: Basics} \small
\begin{itemize} 
\item A String is considered a sequence of characters in programming. 
\item In Python, you use square brackets to access individual characters in a string (indexing starts with 0, not 1). \pause
\item What do the following give you? 
\begin{verbatim}
course = "ENGL 516"
letter1 = course[0]
letter2 = course[1]
letterX = course[1.9] 
letterY = course[10] 
letter11 = course[-1]
letter11= course[-5]
letter11 = course[-15]
\end{verbatim} \pause 
\end{itemize}
\end{frame}

\begin{frame}[fragile]
\frametitle{Strings in Python are immutable}
\begin{itemize}
\item What does it mean? - It means that you cannot change individual characters in an existing string. 
\item i.e., if there is a string variable assignment: course = "ENGL 516", writing: letter1 = course[0] is valid, but writing course[0] = "a" is not. \pause
\item (But, checking if course[0] is "a" in some boolean expression like: course[0] == "a" is valid!) \pause
\item This is also valid: \scriptsize
\begin{verbatim}
s = "python"
s = "valid"
\end{verbatim} \normalsize
(What will s have? where did the immutability go?)
\item You can create a new string which is an altered version of the original string.  (e.g., t = s[0]+"acation")
\end{itemize}
\end{frame}

\begin{frame}[fragile]
\frametitle{Traversing through a string}
\begin{verbatim}
course = "ENGL 516"
index = 0
while index<len(course):
    letter = course[index]
    print(letter)
    index = index+1
\end{verbatim}

Short exercise: Write the for loop equivalent of this while loop.
\end{frame}

\begin{frame}[fragile]
\frametitle{Traversing through a string: For Loop}
\begin{itemize}
\item One way:
\begin{verbatim}
course = "ENGL 516"
for i in range(0,len(course)):
    letter = course[i]
    print(letter)
\end{verbatim}
\item Another way:
\begin{verbatim}
course = "ENGL 516"
for i in course:
    print(i)
\end{verbatim}
\end{itemize}
\end{frame}
%Talk about this: Optional reading: "The Joys (and Woes) of the Craft of Programming" by Frederick P.Brooks 
% \url{http://home.adelphi.edu/sbloch/class/adages/joy.html}

\begin{frame}
\frametitle{Exercise on String traversal}
\framesubtitle{from textbook}
Write a while loop that starts at the last character in the string and works its way backwards to the first character in the string, printing each letter on a separate line, except backwards. 
 
\medskip
For example, if I have a string ENGLISH, your program should print: \\
H \\
S \\
I \\
L \\
G \\
N \\
E 
\end{frame}

\begin{frame}[fragile]
\frametitle{Exercise on String traversal:Solution}
\begin{verbatim}
i = len(some_string)
while i>0:
    print(some_string[i-1])
    i = i-1
\end{verbatim}
\end{frame}

\begin{frame}[fragile]
\frametitle{Another exercise on String traversal}
\framesubtitle{from textbook}
Write a function that takes two arguments: a string and a character, and returns the number of times the character occurs in this string as the result. Here is a skeleton for the function definition:
\footnotesize
\begin{verbatim}
def countChar(someString, someChar):
   count = 0
   for anyChar in someString:
       #Write your 2 lines of code here.
   return count

stringInput = input("Enter a string: ")
charInput  = input("Enter a char: ")
print("The number of times ", charInput, "occured in", stringInput,
    "is",countChar(stringInput,charInput))
\end{verbatim}
\end{frame}

\begin{frame}[fragile]
\frametitle{String "Slicing"}
\begin{itemize}
\item Slice is a partial segment from a string, of any length. 
\begin{verbatim}
str="Python programming"
print(string[7:10])
\end{verbatim}
- this gives me "pro"
\item The operator [n:m] returns the part of the string starting from the nth character and up to (but not including) the mth character. 
\end{itemize}
\end{frame}

\begin{frame}
\frametitle{String "Slicing": Practice}
Try these things on the console, and note the output.
\begin{enumerate}
\item string="Python programming"
\item print(string[5:])
\item print(string[:3])
\item print(string[9:9])
\item print(string[:])
\item print(string[9:3])
\item print(string[-1])
\item print(string[::-1])
\end{enumerate}
\end{frame}

\begin{frame} %Directly give an overview here, and start about the terminology later.
\frametitle{Built-in functions for strings}
\framesubtitle{called "Methods"}
Go back to python console, and type the following and observe the result.
\begin{itemize}
\item example="Some Example String"
\item print(example.upper())
\item print(example.lower())
\item print(example.startswith("S"))
\item print(example.endswith("S"))
\item print(example.isdigit())
\item print(example.find("e"))
\item print(example.find("e",5))
\item print(example.find("tri")
\end{itemize}
\end{frame}


\begin{frame}
\frametitle{Assignment 3 Description}
\begin{itemize}
\item Topics: Strings, Loops
\item Deadline: 17th Feb 2018
\item 10\% of your final grade
\item 3 questions (2.5, 2.5, 5)
\item Questions described in the document on Canvas
\end{itemize}
\end{frame}


\begin{frame}
\frametitle{Next Class}
\begin{itemize}
\item Topics: String manipulations continued, and Regular expressions
\item Readings: Chapter 11 in the text book.
\item Mandatory exercise(s): Submit Assignment 1 by this midnight! 
\item Optional: Exercise 5 at the end of Chapter 6 in the textbook.
\item Remainder: Assignment 2 has to be submitted by 25th February. 
\item Assignment 3 is already uploaded, if you want to start early.  
\end{itemize}
\end{frame}
\end{document}
