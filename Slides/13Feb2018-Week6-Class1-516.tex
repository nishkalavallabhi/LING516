\documentclass{beamer}
\usepackage[utf8]{inputenc}

\author[Sowmya Vajjala]{Instructor: Sowmya Vajjala}


\title[ENGL 516X]{ENGL 516X: \\ Methods of Formal Linguistic Analysis}
\subtitle{Semester: Spring '18}

\date{13 Feb 2018}

\institute{Iowa State University, USA}

%%%%%%%%%%%%%%%%%%%%%%%%%%%

\begin{document}

\begin{frame}\titlepage
\end{frame}

\begin{frame}%2minutes
\frametitle{Class outline}
\begin{itemize}
\item Regular Expressions
\item Using regex in Python code
\item Regex in Python: Practice
\item Links to Regex Practice exercises
\end{itemize}
\end{frame}

\begin{frame}
\frametitle{}
\centering
\Large Regular Expressions
\end{frame}

\begin{frame}
\frametitle{Regular Expressions}
\begin{itemize}
\item Regular expressions are used to do do pattern based information extraction from data. 
\item They have their own syntax for doing pattern matching in different ways.
\item They are very useful to process text and manipulate it.
\item Regular expressions in python are in a module called "re" and you can use them in your code once you add a "import re" statement in your program/console.
\item They can simplify a lot of your tasks, but they themselves can be very complicated.
\item pythex.org - is what I will use today to explain the syntax. We will use import re in our code next week.
\end{itemize}
\end{frame}

\begin{frame}
\frametitle{RegEx syntax}
\begin{enumerate}
\item \^\{\} matches the beginning of a line. For example,
\begin{itemize}
\item a pattern \^{}Th matches all lines in a text file that start with Th
\end{itemize}
\item \$ matches the end of a line. For example,
\begin{itemize}
\item a pattern Th\$ matches all lines in a text file that end with Th
\end{itemize}
\item $\backslash s$ matches a white space character
\item $\backslash S$  matches a non-white space character.
\end{enumerate}
%approx 20min
\end{frame}

\begin{frame}
\frametitle{RegEx syntax}
\begin{enumerate}
\item . matches any character
\item * -applies to the immediately preceding character and indicates to match zero or more of the preceding character(s).
\begin{itemize}
\item for example, te* matches all locations where there is a t, te, tt, tete etc.
\end{itemize}
\item + - applies to the immediately preceding character and indicates to match one or more of the preceding character(s).
\begin{itemize}
\item for example, te+ matches all locations where there is a te, tete, tetete etc.
\end{itemize}
\item \{\} is used next to a regular expression to indicate a range of occurrences of that expression. e.g., t\{1,3\} matches: t,tt,ttt. 
\end{enumerate}
\end{frame}

\begin{frame}
\frametitle{RegEx syntax - continued}
\framesubtitle{The power of square brackets}
\begin{enumerate}
\item \lbrack aeiou\rbrack - matches a single character as long as the character is in this set.
\item You can also specify ranges in square brackets. For example, [a-z0-9] matches all characters in lower case or a single digit.
\item When the first character after the square brackets is a caret (\^{}), it works like a "not" keyword. So, [\^{}a-z0-9] matches all characters that are not lower cased letters, and not numbers.
\end{enumerate}
%approx 20min
\end{frame}

\begin{frame}
\frametitle{Escape Character}
What do you do if you want to match a ? or a . which also carry a meaning in regex? \pause
\\ We "escape" them to tell regex module that these are real characters and not regex syntax. This is done using a $\backslash$ character. 

 \medskip So, st$\backslash$. is a pattern that searches for all occurences of "st." in a string.
%approx 20min
\end{frame}

\begin{frame}
\frametitle{Regex practice on \url{http://pythex.org}}
Go to APLING program homepage (apling.engl.iastate.edu) and copy the welcome message there into pythex test string area. Now, try to write regex patterns to get the following:

\begin{enumerate}
\item All occurences of the word "is" (Not th\textbf{is}, lingu\textbf{is}tics, etc. Only "is") - $\backslash$bis$\backslash$b
\item All occurences of the letter e, irrespective of the case. - $e|E$
\item All occurences of "es" where it occurs in the middle of the word (i.e., es should not be followed by a space, comma, fullstop etc) - [a-z]es[a-z]
\end{enumerate}
\end{frame}

\begin{frame}[fragile]
\frametitle{RegEx: recap exercise}
\begin{itemize}
\item What is the regular expression to select all phone-numbers from the following text:
\begin{verbatim}
University Assistance
515-294-4428	Department of Public Safety
515-294-3322	Department of Residence
515-294-5100	Facilities Planning and Management
515-294-4444	Help Van
Poison Control
800-222-1222	Iowa Poison Control Center
\end{verbatim} 
(source: \url{http://info.iastate.edu/emergency/}) \pause
\item My expression: ([0-9]+-*)+ \pause
\item Slightly complex, but more precise one: ([0-9]\{3\}-)\{2\}[0-9]\{4\}
\end{itemize}
\end{frame}

\begin{frame}
\frametitle{Using RegEx in Python}
\framesubtitle{Python's "re" module}
\begin{itemize}
\item You should import "re" module of python using the statement: import re
\item Two useful functions in re module are: search() and findall().
\item search() in re module is similar to the find() method for Strings, but just more sophisticated.
\begin{itemize}
\item re.search("XX[0-9]",str) searches for the first occurence of "XX" followed by a digit in a string and returns the corresponding match.
\end{itemize}
\item findall() returns all matches of a pattern in a string, as a list of matches.
\end{itemize}
\end{frame}

\begin{frame}
\frametitle{Searching a text with RegEx}
\framesubtitle{re.search() function}
An example code: RegexSearchExamples.py 
\end{frame}

\begin{frame}
\frametitle{A short detour: What is a List?}
\begin{itemize}
\item A list is a sequence of values. A string can be seen as a list of characters. There can be a list of strings as well.
\item These values in a list are called "elements" or "items".
\item Lists in Python are identified by the presence of square brackets. 
\item Examples:
\begin{enumerate}
\item \lbrack 516,515,520,540\rbrack - is a list of integers
\item \lbrack"Python","Java","Perl","R","Ruby"\rbrack - is a list of strings.
\item \lbrack'spam', 2.0, 5, \lbrack10, 20\rbrack\rbrack - is list with elements of various data types. There is a list inside a list too!
\end{enumerate}
\end{itemize}
\end{frame}

\begin{frame}[fragile]
\frametitle{An Example list and its use}
Look at this code:
\begin{verbatim}
str = "Look at this code"
demoList = str.split(" ") #space is here the "delimiter".
#Splits the string wherever there is a space.
print(demoList)
['Look', 'at', 'this', 'code']
print(len(demoList) 
# prints the number of items in a demoList object. 4 here.
print(demoList[1]) 
# prints "at".
\end{verbatim}

More on lists on thursday.
\end{frame}

\begin{frame}
\frametitle{Extracting information from a text using RegEx}
\framesubtitle{re.findall() function}
Example of findall() and a comparison with search() - Example codes on Canvas.
\end{frame}

\begin{frame}[fragile]
\frametitle{What is this pattern doing? -1}
\begin{verbatim}
if re.search('^X\S*: [0-9.]+', line):
    print(line)
\end{verbatim}
\pause
\end{frame}

\begin{frame}[fragile]
\frametitle{What is this pattern doing? -2}
\begin{verbatim}
x = re.findall('^Details:.*rev=([0-9]+)', line)
\end{verbatim}
\end{frame}

\begin{frame}[fragile]
\frametitle{Work with this program}
Download grepProgram.py and mbox.txt files from Canvas. Using this program, find out the number of times did:
\begin{itemize} 
\item source@collab.sakaiproject.org receive emails.
\item source@collab.sakaiproject.org appear in the text anywhere. 
\item email addresses from the domain iupui.edu
\item Example interaction with the program:
\end{itemize}
\begin{verbatim}
Enter a regular expression: ^Author
mbox.txt had 1798 lines that matched ^Author
\end{verbatim}
\end{frame}

\begin{frame}
\frametitle{Resources for PythonRegex}
\begin{enumerate}
\item For learning:
\begin{itemize}
\item Python Docs link \url{https://goo.gl/TTunhz}
\item \url{http://regexone.com/references/python}
\end{itemize}
\item For practice:
\begin{itemize}
\item \url{http://pythex.org/}
\item \url{https://regex101.com/\#python}
\item \url{http://www.pyregex.com/}
\item \url{https://txt2re.com/}
\end{itemize}
\end{enumerate}
\end{frame}

\begin{frame}
\frametitle{Next Class}
\begin{itemize}
\item Topics: Lists in python
\item Readings: Chapter on Lists in the textbook
\item Do: The other exercise at the end of the chapter on regular expressions.
\end{itemize}
\end{frame}
\end{document}
