\documentclass{beamer}
\usepackage[utf8]{inputenc}

\author[Sowmya Vajjala]{Instructor: Sowmya Vajjala}


\title[ENGL 516X]{ENGL 516X: \\ Methods of Formal Linguistic Analysis}
\subtitle{Semester: Spring '18}

\date{30 January 2018}

\institute{Iowa State University, USA}

%%%%%%%%%%%%%%%%%%%%%%%%%%%

\begin{document}

\begin{frame}\titlepage
\end{frame}

\begin{frame}%2minutes
\frametitle{Class outline}
\begin{itemize}
\item Week 3 Recap
\item Loops in Python
\item Loops: Practise exercise
\end{itemize}
\end{frame}

\begin{frame}%3-4minutes
\frametitle{}
\centering Week 3 Recap
\end{frame}

\begin{frame}%3-4minutes
\frametitle{Topics discussed}
\begin{itemize}
\item Boolean and logical expressions
\item Conditional statements
\item Exceptions and handling them
\item Writing your own functions
\end{itemize}
\end{frame}

\begin{frame}
\frametitle{Recap questions}
\begin{itemize}
\item What is the difference between try/Except and if/else? \pause
\item If I ask you to write a program that randomly prints numbers between 1 to 100, how will you approach it? \pause
\item Why do you think we need to write functions? \pause
\item What does a return statement do? Why do we need it? \pause
\item What is the difference between a return statement and a print statement?
\end{itemize}
\end{frame}

\begin{frame}
\frametitle{Terminology} %10 min
\begin{itemize}
\item Function definition: The process of defining/creating a function
\item Function parameters: variables inside the function definition, that can be used as input variables. 
\item Function call: The process of calling a function that was defined before.
\item Function arguments: The variables or values that we pass as input to a function.
\item Function object: The "data type" of a function call. 
\item Return value: Value returned by the function.
\item Void function: A function that does not return anything. 
\end{itemize}
\end{frame}

\begin{frame}[fragile]
\frametitle{Terminology: Examples}
\begin{enumerate}
\item Consider the following: 
 \begin{verbatim}
def example_print_function(some_string):
   print(some_string)
  \end{verbatim}
This is a function definition. some\_string is a parameter. This function does not return anything. So, it is a void function. 
\pause
\item Consider this statement: maximum = max(1,2,3,4,5) \\
This is a function call. 1,2,3,4,5 are the arguments. max(1,2,3,4,5) is a function object. maximum is the return value.
\end{enumerate}
\end{frame}

\begin{frame}
\frametitle{Some rules for using functions in your code}
\begin{itemize}
\item A function needs to be always first defined, before being called.
\item A function definition starts with a def keyword. 
\item A function need not necessary have a return statement, but where you can use return instead of print, use it.
\item Indentation and syntax needs to be strictly followed even with functions.
\end{itemize}
\end{frame}

\begin{frame}
\frametitle{Solution to last class' exercise}
Question: Expand the ctof, ftoc functions program to add the following conditions
\begin{itemize}
\item If the user chooses C and then enters a temparature beyond the range: [-58, +58] Celsius, print a message asking them to enter something realistic and stop. Else, do the conversion.
\item If the user chooses F and then enters a temparature beyond the range: [-130, +130] Fahrenheit, print a message asking them to enter something realistic and stop. Else, do the conversion.
\end{itemize}
Solution: ThursdayHW.py in Module:Week4 on Canvas
\end{frame}

\begin{frame}%10min
\frametitle{}
\centering Loops in Python
\end{frame}

\begin{frame}%10min
\frametitle{Iterations and loops: What and Why?}
\begin{itemize}
\item What is a loop? \pause
\item What is iteration? \pause 
\item Where are these concepts useful? \pause 
\end{itemize}
\end{frame}

\begin{frame}
\frametitle{An example loop flowchart}
\includegraphics[width=0.6\textwidth]{Loops-Eg1.png}
\end{frame}

\begin{frame}
\frametitle{A flowchart with a loop in a loop}
\includegraphics[width=0.5\textwidth]{Loops-Eg2.png}
\end{frame}

\begin{frame}
\frametitle{A flowchart with an infinite loop}
\includegraphics[width=0.6\textwidth]{Loops-Eg3.png}
\end{frame}

\begin{frame}[fragile]%3-4minutes
\frametitle{Loops in Python: a "while" Loop}
\begin{verbatim}
i=0
while i<3:
    print(i)
    i = i+1
print("Done with the while loop!")
\end{verbatim}
\end{frame}

\begin{frame}[fragile]%3-4minutes
\frametitle{Loops in Python: a "for" Loop}
\begin{verbatim}
i=0
for i in range(0,3):
    print(i)
    i=i+1
print("Done with the for loop!")
\end{verbatim}
(There is also another way of looping using a "for". We will get to that next week.
\end{frame}

\begin{frame}[fragile]
\frametitle{Example Program with While loop}
\begin{verbatim}
def someFunction(number):
    result = 1
    i = 1
    while i<=number:
        result = result*i
        i = i+1
    return result
print(someFunction(5)) 
\end{verbatim} 
\end{frame}

\begin{frame}[fragile]
\frametitle{Same program with For loop}
\begin{verbatim}
def someFunction(number):
    result = 1
    for i in range(1,number+1):
        result = result*i
    return result
print(someFunction(5)) 
\end{verbatim} 
\end{frame}


\begin{frame}[fragile]
\frametitle{Infinite Loops}
\begin{itemize}
\item Example 1:
\begin{verbatim}
tempstring = "whatever"
while tempstring == "whatever":
    print(tempstring)
print("You will never see this message")
\end{verbatim}

\item Example 2:
\begin{verbatim}
while True:
   print("I won't stop!")
print("You will never see this message")
\end{verbatim}
\end{itemize} 
\bigskip - these kind of loops will never stop until you apply force (on your keyboard, that is).
\end{frame}

%Break statement.
\begin{frame}[fragile]%3-4minutes
\frametitle{Even such patterns can be put to use!}
\framesubtitle{break statement in Python}
Suppose you want to keep taking input from the user until the user enters "done"
\begin{verbatim}
while True:
    line = input('Enter something: ')
    if line == 'done':
        print("Stopping here")
        break #break statement breaks the loop.
    else:
        print(line)
print('Done!')
\end{verbatim}
\end{frame}

\begin{frame}[fragile]
\frametitle{continue statement in python}
Break lets you out of the loop completely. continue just comes out of current iteration and goes to the next iteration of the loop.
\small \begin{verbatim}
while True:
    line = input("Enter something: ")
    if line == 'pass':
       print("I am passing without printing what you entered")
       continue 
    elif line == 'done':
       print("I am stopping the program.")
       break
    else:
       print(line)
print("Done!")
\end{verbatim}
\end{frame}

\begin{frame}
\frametitle{Choosing between a for and while}
\begin{itemize}
\item Use a for loop if you know, before you start looping, the maximum number of times that you?ll need to execute the body. 
\item  if you are required to repeat some computation until some condition is met, and you cannot calculate in advance when (of if) this will happen, use while.
\item Anything implemented in for, can have a while counterpart and vice versa.
\item My preference: for over while (because we wont get into infinite loops by mistake)
\end{itemize}
\end{frame}

\begin{frame}[fragile]
\frametitle{Exercise: Spot the bug}
\begin{verbatim}
def someFunction(number):
    result = 1
    i = 1
    while i<=number:
        result = result*i
        number = number+1
    return result
print(someFunction(5)) 
\end{verbatim} 
\end{frame}

\begin{frame}[fragile]
\frametitle{Exercise: analyze this}
\begin{verbatim}
def seq3np1(n):
    while n != 1:
        print(n, end=", ")
        if n % 2 == 0:        # n is even
            n = n // 2
        else:                 # n is odd
            n = n * 3 + 1
    print(n, end=".\n")
\end{verbatim}
What will seq3np1(16) print? \pause
16, 8, 4, 2, 1
\end{frame}

\begin{frame}[fragile]
\frametitle{Programming Exercise}
\begin{itemize}
\item Ask the user to enter a number first (integer). Assign it to a variable n.
\item Now, take input from the user n number of times after this. These have to be numbers.
\item Once the input taking is done, you have print the following back to the user: sum of the these numbers, and average. 
\item Example interaction with your program:
\tiny
\begin{verbatim}
> Enter the number of numbers you want to enter: 
5
> Enter a number: 2
> Enter a number: 6
> Enter a number: 5
> Enter a number: 3
> Enter a number: 8
> The sum of these numbers is: 24
> The average of these numbers is: 4.8
\end{verbatim}
\small \item Assume for now that the user is following your directions, and there are no errors to handle.
\end{itemize}
\end{frame}

\begin{frame}
\frametitle{}
\centering Solution Discussion
\\ \tiny (Solution is uploaded after the class as: LoopQuestion.py)
\end{frame}

\begin{frame}
\frametitle{Extending this program}
Add exception handling to this program, to address the following conditions:
\begin{itemize}
\item n is a integer between 2 and 100
\item Each subsequent number is a integer in the range 0 to 10000
\item If the user enters a string or floating point or any non-integer for n, print an error message using try and except and stop.
\item If the user enters anything other than a number after n, detect their mistake using if and else, and print an error message and move on to take next input number.
\item Note: This is Similar to Final exercise in Chapter 5 in the textbook
\item Note 2: I am not asking you to organize this program into functions - but think if you can. 
\end{itemize}
\end{frame}

\begin{frame}[fragile]
\frametitle{Additional Exercise for fast programmers}
\begin{itemize}
\item Write a program that takes a number and prints multiplication table for that number (n*1 to n*10, one number per line)
\item Expected input/output:
\begin{verbatim}
> Enter a number: 5
> 5*1 = 5
  5*2 = 10
  5*3 = 15
  .. ... 
  5*10 = 50
\end{verbatim}
\end{itemize}

\end{frame}

\begin{frame}
\frametitle{Next Class}
\begin{itemize}
\item Topics: Revision of what we learnt so far
\item writing a main() function - good programming practices
\item ToDo: In the forum post for Revision topics - post the topics you want me to discuss on Thursday. I will accomodate as many requests as possible. 
\item There will also be in-class programming exercises as usual on thursday. 
\item If you want little bit more challenging stuff, look at Chapter 7 in the second textbook: \url{http://openbookproject.net/thinkcs/python/english3e/iteration.html}
\end{itemize}
\end{frame}

\end{document}


