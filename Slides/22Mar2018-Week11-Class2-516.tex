\documentclass{beamer}
\usepackage[utf8]{inputenc}
\usepackage{graphicx}
\usepackage{url}

\author[Sowmya Vajjala]{Instructor: Sowmya Vajjala}

\title[ENGL 516X]{ENGL 516X: \\ Methods of Formal Linguistic Analysis}
\subtitle{Semester: Spring '18}

\date{22 March 2018}

\institute{Iowa State University, USA}

%%%%%%%%%%%%%%%%%%%%%%%%%%%

\begin{document}

\begin{frame}\titlepage
\end{frame}

\begin{frame}
\frametitle{Outline for today's class}
\begin{itemize}
\item studying code with database + UI
\item Reminder: Submit Assignment 5 on time. 
\end{itemize}
\end{frame}

\begin{frame}%5min
\frametitle{Review of Tuesday's Class}
\begin{enumerate}
\item I am assuming you went through those code examples \pause
\item If I now ask give a table description, and ask you to write python code to create such a table, what operation should you perform? \pause
\item How will you write a query to choose all rows in a table, where the Column "role" has the value "faculty"? \pause
\end{enumerate}
\end{frame}

\begin{frame}
\frametitle{}
\centering
\Large Let us take a look at the three code files from Tuesday (Install: sqlite browser (http://sqlitebrowser.org/).
\end{frame}

\begin{frame}
\frametitle{}
\centering
\Large Okay, now, be brave.
\end{frame}

\begin{frame}
\frametitle{How do you teach someone to swim?}
\begin{itemize}
\item Let us say my answer is: "I will just push this person into water. They will learn to swim." - what would be your reaction?
\pause \item It is a somewhat extreme form of teaching, dangerous too in this case. 
\pause \item Now, am going to follow this kind of approach (and it is not physically dangerous).
\end{itemize}
\end{frame}

\begin{frame}
\frametitle{What is it?}
\begin{itemize}
\item Some time back, I did a web-based reading study, where participants came, logged into the application, and read some texts and answered questions about it.
\item The entire setup was coded in Python, with Bottle, and with Sqlite3 module.
\item Code was written by Stefanie Fuccio, a student in my 516 in Spring 2016. \pause
\item Stephanie started working on this immediately after 516 (she had no prior background), and while taking 520.
\item I guess it took around 3 months for her to make it work, do some testing and stuff (For experienced folks, it takes perhaps 3-4 weeks max). 
\item Let me show you how it works.
\end{itemize}
\end{frame}

\begin{frame}
\frametitle{Your task -1}
\begin{itemize}
\item The code, and the sqlite file are all in a zip file with today's date on Canvas.
\item Form into groups of two or three people, look at one of: 2\_createlogins.py and 3\_createtexttable.py (people on left, take 2\_, on right, take 3\_, middle - up to you)
\item You have 20 minutes. After that, I will "randomly" pick people to come and explain these two programs. 
\end{itemize}
\end{frame}

\begin{frame}
\frametitle{Next Week}
\begin{itemize}
\item Assignment 6 introduction, Database-UI discussion continuation
\item Working with external python modules, other people's code etc. 
\item ToDo: Understand what is happening in this code, relative to how it is progressing on the web application. 
\item Usernames and passwords you need can be accessed from the logins table using sqlite browser.
\end{itemize}
\end{frame}



\end{document}

%in third class, UI + Databases combo examples.
