\documentclass{beamer}
\usepackage[utf8]{inputenc}
\usepackage{graphicx}
\usepackage{url}

\author[Sowmya Vajjala]{Instructor: Sowmya Vajjala}


\title[ENGL 516X]{ENGL 516X: \\ Methods of Formal Linguistic Analysis}
\subtitle{Semester: Spring '18}

\date{27 March 2018}

\institute{Iowa State University, USA}

%%%%%%%%%%%%%%%%%%%%%%%%%%%

\begin{document}

\begin{frame}\titlepage
\end{frame}

\begin{frame}
\frametitle{Class Outline}
\begin{itemize}
\item Assignment 5 discussion
\item Assignment 6 description
\item Working with webpages
\item Working with external python libraries: case study with BeautifulSoup
\end{itemize}
\end{frame}

\begin{frame}
\frametitle{Guiding questions for this week}
\begin{itemize}
\item How to read information from different urls on the web?
\item How to parse HTML (or other such structured text, eg. XML) without writing complicated regular expressions? (1 class)
\item How do I read other people's code and make sense out of it? (continuation of last week)
\end{itemize}
\end{frame}

\begin{frame}
\frametitle{Assignment 5 discussion}
\begin{itemize}
\item First question -straight forward if you follow the instructions in the book
\item Second question - not too difficult if you are comfortable with loops and conditionals
\item Third question - slightly more challenging, only 2 groups attempted, I think (and 1 group seems to have got it fully correct)
\end{itemize}
I need volunteers to discuss their solutions (all people in the group should participate)
\end{frame}

\begin{frame}
\frametitle{Assignment 6 description}
\begin{itemize}
\item Group assignment
\item Grade points: 10
\item Deadline: 07 April 2018
\item Task: combination of bottle + sqlite3
\item You have all you need in the examples discussed in classes
\end{itemize}
\end{frame}

\begin{frame}
\frametitle{Reading from a webpage}
\framesubtitle{Python's urllib library}
\begin{itemize}
\item urllib library in python is a collection of Python modules which allow us to read content from webpages.
\item It has four modules: 
\begin{enumerate}
\item urllib.request - to open urls, read stuff from them
\item urllib.error - deals with the errors arising out of the open/read process
\item urllib.parse - for parsing urls. Inside a url, answers questions like: is this http or https? what are the parameters (e.g., xyz.com?user=a\&name=b) or queries, and so on. 
\end{enumerate}
\item Good intro and tutorial on urllib: \url{https://docs.python.org/3/library/urllib.html}
\end{itemize}
\end{frame}

\begin{frame}[fragile]
\frametitle{Reading from a webpage, line by line}
\footnotesize
\begin{verbatim}
import urllib.request
fhand = urllib.request.urlopen('http://www.py4inf.com/code/romeo.txt')
for line in fhand:
   print(line.decode(encoding="utf8"))
\end{verbatim}
\end{frame}

\begin{frame}[fragile]
\frametitle{Reading from a webpage -2}
What if the webpage is not a text file.. and it is a general HTML file?
\tiny
\begin{verbatim}
import urllib.request
url = "http://www.theunixschool.com/2012/09/examples-how-to-change-delimiter-of-file-Linux.html"
fhand = urllib.request.urlopen('url')
for line in fhand:
   print(line.decode(encoding="utf8").strip())
\end{verbatim}
\normalsize This will print you all the html, and you need to write your regex or use other means to extract the text you want.
\end{frame}

\begin{frame}[fragile]
\frametitle{What if the url is an image?}
\scriptsize
\begin{verbatim}
import urllib.request
imagepath = "http://www.phdcomics.com/comics/archive/phd031813s.gif"
fhand = urllib.request.urlopen(imagepath)
for line in fhand:
   print(line)
\end{verbatim}
What will happen now?
\end{frame}

\begin{frame}[fragile]
\frametitle{So how should I "download" the image?}
\scriptsize
\begin{verbatim}
import urllib.request
imagepath = "http://www.phdcomics.com/comics/archive/phd031813s.gif"
fhand = urllib.request.urlretrieve(imagepath)
\end{verbatim}
Now, fhand will tell you where the image is in our computer.
What will happen now?
\end{frame}

\begin{frame}[fragile]
\frametitle{Download and Save a image}
\footnotesize
\begin{verbatim}
img = urllib.request.urlopen('http://www.py4inf.com/cover.jpg').read()
fhand = open('cover.jpg', 'wb')
fhand.write(img)
fhand.close()
\end{verbatim}
\end{frame}

\begin{frame}
\frametitle{"Scraping" HTML}
\begin{itemize}
\item One of the popular uses of urllib library is to extract text from webpages. This is called "scraping".
\item Scraping is also what a search engine like google does, when it crawls all the webpages, and then, retrieves them when we query it.
\item So, how do we scrape? One simple (not really simple) way: regular expressions. \pause
\item We saw some regular expression HTML examples in the past few weeks \pause
\item Another way is to use one of the several python libraries for HTML parsing.
\end{itemize}
\end{frame}

\begin{frame}
\frametitle{Scraping HTML with Regular Expressions}
Three important things to know:
\begin{enumerate}
\item How can I look at the html source of a webpage? \pause
\item How can I find what patterns I should target? \pause
\item Third, but most important: how to write regular expressions!
\end{enumerate}
\end{frame}

\begin{frame}
\frametitle{Scraping HTML with Python Libraries}
\begin{itemize}
\item There are a lot of Python libraries to do HTML scraping and parsing.
\item Beautiful Soup is one of them, and very popular (It was popular in 2010 too, when I first used it)
\item Lot of HTML on the web is broken.. but Beautiful Soup is very tolerant to such broken HTML and still gives you clean output.
\item How to install? - install from PyCharm. Go the DIY way and figure out how to install.
\end{itemize}
\end{frame}

\begin{frame}
\frametitle{BeautifulSoup}
\framesubtitle{Installation}
\begin{itemize}
\item How many of you successfully installed Beautiful Soup on your machines? \pause
\item \url{https://www.youtube.com/watch?v=HJ9bTO5yYw0} - a video for doing it on PyCharm. 
\item only difference: in Python3, after installation, you write: "from bs4 import BeautifulSoup"
\end{itemize}
\end{frame}

%Theme: Beautiful soup
\begin{frame}
\frametitle{Using Beautiful Soup}
BeautifulSoupBasics.py
\end{frame}

\begin{frame}
\frametitle{How to work with Beautiful Soup}
\url{http://www.crummy.com/software/BeautifulSoup/bs4/doc/}
\end{frame}

\begin{frame}
\frametitle{Reading XML in Python}
\begin{itemize}
\item After HTML, another common format of saving textual information is XML. 
\item XML is a structured markup, that is sometimes used to save corpora, database containing multiple fields per text etc.,
\item Sometimes, it is also used by some programs that have a web query interface, to transmit results to another program. 
\item So, learning to parse XML gives you two benefits: to use corpora stored in xml format easily, and to make use of the API of some programs, so that we can build on their output.
\end{itemize}
\end{frame}

\begin{frame}
\frametitle{Some example XML files}
\begin{itemize}
\item Using XML to store a corpus (On browser: xml-example.xml, xml-example2.xml)
\item Using XML to send a response over internet (On browser: Language Tool's output) \\ \url{https://languagetool.org:8081/?language=en-US&text=my+texd}
\end{itemize}
\end{frame}

\begin{frame}
\frametitle{Parsing XML in Python}
ParsingXMLExample.py, ParsingXMLExample2.py - in Canvas.
\\ One uses Python's XML parser, Another example uses Beautiful Soup
\\ More examples on parsing xml in python at: \url{http://www.diveintopython3.net/xml.html}
\end{frame}

\begin{frame}
\frametitle{Practice Exercises}
 Write a program to read in any wikipedia page on some topic (eg: Python programming language), and get all other language pages related to that webpage as the output. You can use Beautiful Soup, or Regular Expressions. 
 If you see a access forbidden error while trying to read a page, figure out how to fix it. 
\end{frame}

\end{document}
