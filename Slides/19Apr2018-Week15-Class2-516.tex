\documentclass{beamer}
\usepackage[utf8]{inputenc}
\usepackage{graphicx}
\author[Sowmya Vajjala]{Instructor: Sowmya Vajjala}

\title[ENGL 516]{ENGL 516: Methods of Formal Linguistic Analysis}
\subtitle{Semester: Spring '18}

\date{19 Apr 2018}

\institute{Iowa State University, USA}
%%%%%%%%%%%%%%%%%%%%%%%%%%%

\begin{document}

\begin{frame}\titlepage
\end{frame}

\begin{frame}
\frametitle{Class Outline}
\begin{itemize}
\item Assignment 7 discussion
\item About project presentations next week
\item About Final submission
\item Using other people's code: exercise
\end{itemize}
\end{frame}

\begin{frame}
\frametitle{}
Assignment 7 discussion
- one volunteer per question
\end{frame}

\begin{frame}
\frametitle{Project presentations schedule next week}
\begin{itemize}
\item Tuesday, 24th: Brody, Emily; Kristin, Afnan; Tim, Taichi, Fatemeh
Fatemeh
\item Thursday, 26th: Sondoss, Hardi, William; Yasin, Haeyun, Nazli
\item Time: 15 min per team (+ time for questions)
\item Expectation: Show the working part, discuss code, talk about how it can be improved, etc. 
\end{itemize}
\end{frame}

\begin{frame}
\frametitle{Final Submission}
\begin{itemize}
\item deadline: 1st May 2018
\item what it should contain: your code, preferably with some comments; a short report describing what you did, how you can improve it if you had time or more expertise, what did you learn from the project etc. 
\end{itemize}
\end{frame}

\begin{frame}
\frametitle{Info from a past student, Stephanie Fuccio}
\begin{itemize}
\item Visual Programming with bubble.is
\item Idea: Build software, but with less traditional coding experience.
\item Stephanie's comments: "So, with this button type of programming I should be able to make my app with less technical expertise (and less head hitting wall moments) while still using the programming logic for the workflow and such, so I will be using the knowledge but not the skill, hmmmm)"
\item Her idea: to use this to build some language learning/testing apps. 
\end{itemize}
\end{frame}

\begin{frame}
\frametitle{Problem Solving: practice}
\begin{itemize}
\item Goal: I want you to think through the problem solving process, and draw an outline of your approach
\item You will have about 15 minutes to first think through the solution
\item I will ask someone to talk about their approach (use whiteboard) after 15 min.
\item You can write code too, if you finish this process and have time later. 
\end{itemize}
\end{frame}

\begin{frame}
\frametitle{Idea: Test of general knowledge}
\begin{itemize}
\item I am uploading a csv file states.csv, which is a spreadsheet with a few text columns (you can read it normally with open() and then split each line by comma to get the strings)
\item make a database table out of this information
\item Program should have a UI, which RANDOMLY selects one row from the table, and presents some questions to the user.
\item Then, it compares user answers to the actual answers, and tells the user how many he got right.
\end{itemize}
\footnotesize source for data: \url{http://www.ipl.org/div/stateknow/chart.html}
\end{frame}

\begin{frame}
\frametitle{Next task: Lexical Complexity Analyzer}
\begin{itemize}
\item Read the code, understand what is happening (it is not extremely commented or anything)
\item You can read the documentation, run it to check if it works for a single file, a folder of text files etc. Understand what the output is. 
\item My question: How are the parts of speech getting calculated for words in this code?
\item Post your answers in the forum for today.
\end{itemize}
\end{frame}

\begin{frame}
\frametitle{Next Week}
your presentations!! 
\end{frame}

\end{document}
