\documentclass{beamer}
\usepackage[utf8]{inputenc}

\author[Sowmya Vajjala]{Instructor: Sowmya Vajjala}


\title[ENGL 516X]{ENGL 516X: \\ Methods of Formal Linguistic Analysis}
\subtitle{Semester: Spring '18}

\date{20 Feb 2018}

\institute{Iowa State University, USA}

%%%%%%%%%%%%%%%%%%%%%%%%%%%

\begin{document}

\begin{frame}\titlepage
\end{frame}

\begin{frame}%2minutes
\frametitle{Class outline}
\begin{itemize}
%\item Announcement: Tutorial date and time
\item Assignment 3 Discussion
\item Assignment 4 Description
\item Lists recap questions
\item Tips for using lists
\item Reading and Writing files in Python
\item Resource for visualizing programs: \url{http://pythontutor.com}
\end{itemize}
\end{frame}

\begin{frame}
\frametitle{Assignment 3 Discussion}
\begin{itemize}
\item Volunteer for discussing Q3 \pause
\item String method .count() - why it won't work here: "baabaaaaa".count("aa")
\end{itemize}
\end{frame}

\begin{frame}
\frametitle{Assignment 4 Description}
\begin{itemize}
\item Deadline: 10th March (changed from 3rd March)
\item 10\% of your grade (last individual assignment)
\item 2 questions: This covers parts of what you learnt before (regular expressions) and what we will cover in this week
\item This is actually quite difficult compared to Assignments 1--3. So, start early.
\item Note: Deadline is 10th March, but I am unavailable (even on email) after 8th.
\end{itemize}
\end{frame}

\begin{frame}[fragile]
\frametitle{Lists recap-1}
\framesubtitle{What will the following print?}
\begin{verbatim}
def funct1(lst1):
   lst1.sort()

def funct2(lst1):
   lst2 = lst1[:]
   lst2.sort(reverse=True)
   return lst2

lst1 = [9,8,7,12,1,2]
print(funct1(lst1))
print(lst1)
print(funct2(lst1))
print(lst1)
\end{verbatim}
\end{frame}


\begin{frame}[fragile] %5min
\frametitle{Lists Recap-2}
\framesubtitle{let us analyse this code}
\scriptsize
\begin{verbatim}
string = input("Enter a paragraph of text:\n")
splitString = string.split(" ")
varN = 5
result = ""
for loopvar in range(0,len(splitString)):
    if loopvar == varN and varN < len(splitString):
        result = result + " " + "____"
        varN = varN +5
    else:
        result = result + " " + splitString[loopvar]
#Question: What does result have now?
print(result)
\end{verbatim}
Use a test string: "What is happening here? What is this code doing?"
\end{frame}

\begin{frame}%7min
\frametitle{Tips for using Lists - 1}
\begin{enumerate}
\item Be aware that some of the list methods manipulate the list and do not return anything lst1.sort() sorts the list lst1 directly..and not save the result as a new list! \pause
\item There are multiple ways of doing the same thing with minor differences (+ vs append, l1.sort() vs sorted(l1) etc.). Pick one idiom and stick to that to avoid confusion.\pause
\item Make a copy of the list before you use functions like sort(), to avoid aliasing. \pause
\item It is easy to fall to write a code that will break easily, especially when we are using regular expressions, split() etc. So, be careful (and patient). Handle possible exceptions, use print statements to do some debugging.
\end{enumerate}
\end{frame}

\begin{frame}[fragile]
\frametitle{Tips for using lists - 2}
What is a potential problem with this code?
\\ (Ignore the first two lines. My question is related to the rest of the code.)
\scriptsize
\begin{verbatim}
fhand = open('mbox-short.txt')
for line in fhand:
    words = line.split()
    if words[0] != 'From': 
        continue
    print(words[2])
\end{verbatim}
\end{frame}

\begin{frame}[fragile]
\frametitle{File processing in Python}
\begin{itemize}
\item There are three basic operations involved: opening a file, reading its contents, and writing content into a new or existing file.
\item We will be talking primarily about files with text content in this course.
\item Why should we know file processing?: Several advantages. We can do corpus analysis easily, for example.
\item Knowing how to use files along with regular expressions is sufficient to do a lot of text processing even if there are no additional any additional language tools (like parsers, taggers etc).
\end{itemize}
\end{frame}

\begin{frame}[fragile]
\frametitle{Opening a file}
\begin{itemize}
\item You open a file with open() function. open("/home/Desktop/a.txt") creates a "handle" for the file if the file really exists on your computer. Otherwise, it throws an error.
\item Assuming that I have mbox-short.txt, what will happen if I type this?
\scriptsize
\begin{verbatim}
fhand = open('mbox-short.txt') 
#this looks for the file in your current folder.
print(fhand)
\end{verbatim} \pause \normalsize
\item Assuming that I don't have a file that i am trying to open, what will happen?
\scriptsize
\begin{verbatim}
fhand = open('some_random_name.txt') 
print(fhand)
\end{verbatim} \normalsize \pause
\item One way to avoid such errors is to use try-except loops while trying to read data from files.
\end{itemize}
\end{frame}

\begin{frame}[fragile]
\frametitle{Reading a file}
\begin{itemize}
\item There are two ways to read a file in python. Read line by line (makes more sense for a text file). Read as a whole (useful for text files, and also other formats).
\item Reading line by line:
\begin{verbatim}
fhand = open('mbox-short.txt')
content = "" 
for line in fhand:
  content = content + line
\end{verbatim}
\item Reading the whole file at once:
\begin{verbatim}
fhand = open('mbox-short.txt') 
content = fhand.read()
\end{verbatim}
\end{itemize}
\end{frame}

\begin{frame}[fragile]
\frametitle{Checking if a file really exists}
\begin{itemize}
\item As mentioned earlier, one way is to use try-except loop and see if there is a "FileNotFound" error when you try to read the file. 
\item Other way is to use a built in module os and its isfile() function.
\item example at: CheckFilePath.py 
\end{itemize}
\end{frame}

\begin{frame}[fragile]
\frametitle{Searching through a file}
One example: 
\begin{verbatim}
fhand = open('mbox-short.txt')
for line in fhand:
    line = line.rstrip()
    if line.startswith('From:') :
        print(line)
\end{verbatim}
\end{frame}

\begin{frame}[fragile]
\frametitle{Searching through a file}
One more example:
\begin{verbatim}
fhand = open('mbox-short.txt')
for line in fhand:
    line = line.rstrip()
    if line.find('@uct.ac.za') == -1 : 
        continue
    print(line)
\end{verbatim}
\pause This will print only lines that have that string "@uct.ac.za".
\end{frame}

\begin{frame}
\frametitle{More examples of searching}
.. in regular expressions chapter, lot of practice while doing Assignment 4. 
\end{frame}

\begin{frame}[fragile]
\frametitle{Writing into a file}
\begin{itemize}
\item You still have to "open" a file, but add an extra parameter called "w"
\\ writeHandle = open('output.txt', 'w')
\item You then use the write() method of this handle to write content into the file.
\begin{verbatim}
writeHandle.write("this is a line\n")
\end{verbatim}
\item The handle will not do a "enter" keypress. So, you need to put your own newlines.
\item Once you are done, close the handle. \\ writeHandle.close()
\item Caution: If the file already exists, doing this will delete the old data! so be careful! If the file doesn’t exist, a new one is created.
\end{itemize}
\end{frame}

\begin{frame}[fragile]
\frametitle{Getting a list of all files in the directory}
\begin{itemize}
\item Sometimes, we want to work with a bunch of files instead of a single file. 
\item It is useful to have a way to list all files (or just .txt files or .py files etc.).
\item ...and later we can loop through the directory, process one file after another iteratively.
\item Go to: ListFilesInDir.py
\end{itemize}
\end{frame}

\begin{frame}
\frametitle{Time for a small exercise}
Try to write a small code to read a file given by the user, and give: a pdf file as input, and see what happens. Give it one of your .py files as input and see what happens.
\end{frame}

\begin{frame}
\frametitle{Post solution on the forum}
Try to write a small code that will take a folder/directory path, and lists the number of lines of all files in that directory (if it can actually read it!).
\end{frame}

\begin{frame}
\frametitle{Next Class}
\begin{itemize}
\item Two useful data structures in Python: Dictionaries and Tuples (chapters: 9--10)
\item Next Tuesday: Recap, Practice. So, post any questions you have online in the forum with "Recap: 27 Feb" as the title.
\end{itemize}
\end{frame}

\end{document}
