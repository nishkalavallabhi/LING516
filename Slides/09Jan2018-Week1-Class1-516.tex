\documentclass{beamer}
\usepackage[utf8]{inputenc}

\author[Sowmya Vajjala]{Instructor: Sowmya Vajjala}


\title[ENGL 516X]{ENGL 516X: \\ Methods of Formal Linguistic Analysis}
\subtitle{Semester: Spring '18}

\date{9 Jan 2018}

\institute{Iowa State University, USA}

%%%%%%%%%%%%%%%%%%%%%%%%%%%

\begin{document}

\begin{frame}\titlepage
\end{frame}

\begin{frame}
\frametitle{Class outline}
\begin{itemize}
\item Introductions
\item Motivation for the course and objectives
\item Course logistics
\item Course Outline and Syllabus 
\item Comments from past students
\end{itemize}
\end{frame}

\begin{frame}
\frametitle{}
\begin{center}
\Large Introductions
\end{center}
\end{frame}

\begin{frame}
\frametitle{About me}
\begin{enumerate}
\item In ISU as Asst. Professor since January 2016.
\item Education: PhD in Computational Linguistics, 2015.
\item Teaching experience:
\begin{itemize}
\item ALT students@ISU: 516, 520
\item Other Graduate courses@ISU: 515, 590 (Independent study supervision)
\item Undergrad@ISU: 120, 314, 410
\item Topics seminars for computational linguistics students@UT\"u (2012-13)
\end{itemize}
\end{enumerate}
\end{frame}

\begin{frame}
\frametitle{About you}
\begin{enumerate}
\item Name
\item What do you study in ISU?
\item Why do you want to learn computer programming?
\end{enumerate}
\end{frame}

\begin{frame}
\frametitle{}
\begin{center}
\Large About the course
\end{center}
\end{frame}

\begin{frame}
\frametitle{Course Objectives}
\begin{itemize}
\item Introduce you to the world of computer programming.
\item Make you understand why this is an important skill even for applied linguists in today's world
\item Make you have a problem solving mind set which will be useful in doing your research too
\end{itemize}
\end{frame}

\begin{frame}
\frametitle{Expected Learning Outcomes}
\begin{itemize}
\item Given a problem description, think about an approach to solve the problem, design and write computer program(s) to implement the approach
\item Understand how to find and fix errors in your programs (debugging)
\item Know where to look for to not reinvent the wheel. Read the documentation for existing code and write their own programs extending them.
\item Understand how text is represented on computers, and how to work with it using regular expressions
\item Write small programs that involve reading, processing and writing content into files on the computer
\item Implement simple browser based applications involving collecting and displaying data to the user
\end{itemize}
\end{frame}

\begin{frame}
\frametitle{What will you not become?}
\begin{enumerate}
\item Expert programmer
\item Expert Python programmer
\item Mr/Ms Know it all about computers
\end{enumerate} \pause
\includegraphics[width=0.6\textwidth]{acharyat.png}
\end{frame}

\begin{frame}
\frametitle{Pre-requisities}
\begin{enumerate}
\item familiarity with using computers
\item interest in learning how to program for computers
\item passion about developing and using information technnology for educational applications
\item background in linguistics is useful but not mandatory.
\end{enumerate}
\end{frame}

\begin{frame}
\frametitle{}
\begin{center}
\Large Motivation
\end{center}
\end{frame}

\begin{frame}
\frametitle{Why should I learn programming?}
\begin{itemize}
\item Understanding some of the technology: We use a lot of products that involve programming in everyday life. Your computer and mobile phone are the most used, for example.
\item You can automate some boring, repetitive tasks using a program. (e.g., everyday, at 5pm, take a backup of folder X and save it on the server).
\item You can write your own custom programs to do specific corpus analysis, or identify a specific kind of error in student writing and so on.
\item Programming gives you instant gratification :-)
\end{itemize}
\end{frame}

\begin{frame}
\frametitle{Programming and Applied Linguistics}
\framesubtitle{Some Practical Examples}
\begin{itemize}
\item develop your own modules in CALL systems. 
\item write programs that highlight grammar/spelling mistakes in student writings
\item create test exercises automatically for the class (e.g., writing code to create fill in the blanks questions)
\item Those who want to do user studies: set up our own experiments, to test the phenomenon we want to study (e.g., use of phrasal verbs by English learners of Turkish descent).
\item Programs that analyze the student essays and point of lack of coherence or illogical statements (alright, far fetched.. but still). 
\end{itemize}
... and so on.
\end{frame}

\begin{frame}
\frametitle{}
\begin{center}
\Large Course Logistics
\end{center}
\end{frame}

\begin{frame}
\frametitle{Meeting and Location}
\begin{itemize}
\item  meets in Ross 312, on Tuesdays and Thursdays from 12:40-2 pm
\item \textit{Office hours:} Tuesdays and thursdays, 2-3 pm (please email beforehand if there are specific issues to discuss)
\item course website: on Canvas.
\end{itemize}
\end{frame}

\begin{frame}
\frametitle{Credits}
\begin{itemize}\vspace*{-.8\baselineskip}\itemsep0ex
\item Credit Points: 3
\\ $=>$ (3 hours of classroom instruction + 6 hours of additional effort on student's end). \\
Note: This course is very different from what you are used to in Applied Linguistics. Be prepared to not "get it right" the first time (No one does!)
\end{itemize}
\end{frame}

\begin{frame}
\frametitle{}
\begin{center}
Format and Grading
\end{center}
\end{frame}

\begin{frame}
\frametitle{Course Format}
\begin{itemize}\itemsep2ex
\item weekly lectures/practical sessions, readings
\item participation in discussion forum
\item 7 assignments
\item final exam - programming project.
\end{itemize}
\end{frame}

\begin{frame}
\frametitle{Assignments}
\begin{itemize}
\item 7 Assignments (65\%)
\item Final project (30\%)
\item Classroom participation (5\%)
\end{itemize}
Deadlines and timeline of general events are all in the syllabus document. [Take a look]. First 3 assignments are already uploaded. Rest will be uploaded in the coming 2-3 weeks. Final project descriptions also will be up in a few weeks. 
\end{frame}

\begin{frame}
\frametitle{Syllabus - Topics}
\begin{enumerate}
\item introduction to computing and programming.
\item introduction to python and making python work on your computers
\item basic concepts of writing code in python (variables, functions, recursion etc.)
\item Python datastructures: lists, dictionaries, tuples etc
\item strings and regular expressions
\item reading and writing text from files
\item understanding program flow, using other people's code in your code
\item database basics
\item unix fundamentals
\end{enumerate}
\end{frame}

\begin{frame}
\frametitle{Text Book and Other Material}
\begin{enumerate}
\item Primary textbooks (both are free ebooks)
\begin{enumerate}
\item Python for Informatics by Charles Severence
\item How to think like a computer scientist: Learning with Python3
\end{enumerate}
\item Other useful resources:
\begin{itemize}
\item Free online courses in Coursera by Charles Severence, under "Python for Everybody" specialization series
\item Unix for Poets, free ebook by Kenneth Ward Church
\end{itemize}
\end{enumerate}
... and several other resources online. \\ \medskip

Some additional references on Canvas in readings.pdf file.
\end{frame}

\begin{frame}
\frametitle{Relevance of Assignments}
\begin{itemize}
\item There are two goals for the assignments, and classroom exercises:
\begin{enumerate}
\item Give you more practice with the programming language (learning the correct syntax, using the right vocabulary, teaching you problem solving)
\item Give you some background in working with text corpora  
\end{enumerate}
\pause \item Sometimes, the assignments may look very irrelevant to your own background (e.g., if I ask you to reverse a number i.e., 123 should become 321). But, there is a reason behind asking such a question (practice with different functions in python, how to solve a new problem, etc)
\item \pause So, don't think I am stupid or you are stupid. Don't also start imagining you know better than me about how to teach this particular subject. Irrespective of your tons of experience in some other field, we are here in these roles because I have more experience in this topic. So, trust my decisions.
\item You can ofcourse give me a wishlist of things you want to see in this class. 
\end{itemize}
\end{frame}

\begin{frame}
\frametitle{Some general rules:}
\begin{itemize}
\item attendance: not mandatory, but recommended.
\item deadline extension: with penalty
\item long absence due to illness etc: please inform and follow university procedures.
\item cheating and plagiarism: see the course handbook, and university policy against plagiarism.
\item classroom behavior: please be punctual and do not do personal work in the class.
\item Disability accomodation: Please speak to Disability Resources Office (DRO) to officially request an accomodation.
\item reporting grievances: follow standard procedures. 
\end{itemize}
\end{frame}

\begin{frame}
\frametitle{Any questions so far?}
\end{frame}

\begin{frame}
\frametitle{A small problem solving question}
\framesubtitle{source: hackerrank.com}
Think about this problem described below (work in groups), and try to understand how to solve it: \\

\textit{Davis has staircases in his house and he likes to climb each staircase 1, 2, or 3 steps at a time. Being a very precocious child, he wonders how many ways there are to reach the top of the staircase. Given the number of steps in a stair case, how many different ways can be climb it? Is there a pattern to predict the combinations based on number of steps?}

E.g., 1 step - only 1 way. 2 steps, there are two ways (1 step at a time, or 2 steps at once) etc. 
\end{frame}

\begin{frame}
\frametitle{Comments from old students}
\begin{itemize}
\item Kimberley Becker
\item Jordan Smith
\item Ziwei Zhou
\end{itemize}
\end{frame}

\begin{frame}
\frametitle{ToDo before next class} 
\begin{itemize}
\item Read Chapter 1 in the textbook (Severence).
\item Read the article "The English Teacher as a programmer" (\url{http://computersandcomposition.candcblog.org/archives/v4/4_3_html/4_3_5_Costanzo.html}). Participate in the discussion forum and share your thoughts about it.
\end{itemize}
\end{frame}

\begin{frame}
\frametitle{Next Class ..} 
\begin{itemize}\itemsep2ex
\item Discussion about the readings
\item A brief history of computing
\item How does a computer work?
\item What is a program?
\item TODO: Go through the Canvas website, check the syllabus document, ask questions if any, on the discussion forum.
\end{itemize}
\end{frame}




\end{document}
