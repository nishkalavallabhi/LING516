\documentclass{beamer}
\usepackage[utf8]{inputenc}
\usepackage{graphicx}
\author[Sowmya Vajjala]{Instructor: Sowmya Vajjala}

\title[ENGL 516]{ENGL 516: Methods of Formal Linguistic Analysis}
\subtitle{Semester: Spring '18}

\date{26 Apr 2018}

\institute{Iowa State University, USA}
%%%%%%%%%%%%%%%%%%%%%%%%%%%

\begin{document}

\begin{frame}\titlepage
\end{frame}

\begin{frame}
\frametitle{To Do}
\begin{itemize}
\item Upload your presentations on Canvas by today. I cannot grade you until you submit (10\% of the grade). One submission per group is fine. 
\item Guidelines for final submission:
\begin{itemize}
\item Deadline: May 1, end of the day (20\% of the final grade)
\item What it should contain: All code you wrote, and a report (not exceeding 5 pages) describing your project, what you did with Python, How it works, how can you improve etc.
\end{itemize}
\item Submit your course evals - do them today in the remaining time!!
\end{itemize}
\end{frame}

\begin{frame}
\frametitle{Quick Summary of what we learnt}
In terms of working with textual data, we looked at:
\begin{itemize}
\item Basic setup of Python, variables, expressions, operators etc.
\item Different data structures: strings, lists, tuples, dictionaries etc
\item Functions (built-in, self-written)
\item Installing other python libraries and working with them
\item Making sense of code written by others
\item Web-interfaces
\item Databases
\item Last, but not the least: How to think through the process before writing a program!
\end{itemize}
\end{frame}

\begin{frame}
\frametitle{Far from complete ..}
Remember: We did only a basics course.
\begin{itemize}
\item There is lot more to explore, and there are more things you can with this knowledge within Applied Linguistics.
\item I hope you will continue being interested on this and will enhance your knowledge in future.
\item It is not easy, but it is certainly doable, with some patience, consistent effort, and with a mentor at your workplace in future.
\item Most importantly - don't give up because of bugs in the first version of your program!
\item Second important thing - It is completely alright to write notes/work on a board or on paper before you start typing a program!
\end{itemize}
\end{frame}

\begin{frame}
\frametitle{Something useful I found yesterday}
Python Regular Expression Cheatsheet:  \\ \small
\url{https://www.kdnuggets.com/2018/04/python-regular-expressions-cheat-sheet.html}
\end{frame}

\begin{frame}
\frametitle{Road Ahead: 520 in Fall}
(Based on what I taught)
\begin{enumerate}
\item Topics: Morphological analysis, POS tagging, syntactic parsing, semantics, extracting patterns from corpora etc.
\item Textbooks: NLTK Book (Online and free. Print book does not have Python3 examples) and "Speech and Language processing" by Jurafsky and Martin (not language specific. Good for conceptual knowledge, but also heavy on math). 
\item How much python do you need for 520?: a lot of practice with what you already learnt (or proficiency in some other programming language) and the willingness to quickly learn new concepts and install new tools to try out things.
\item My syllabus at that time: \small \url{http://sowmya.public.iastate.edu/Syllabi/520-Fall16-Syllabus.pdf}
\end{enumerate}
\end{frame}

\begin{frame}
\frametitle{For ideas on programming projects..}
\begin{enumerate}
\item Think about what you want to do and how programming is useful for that goal.
\item Talk to faculty in the department to know what language tools would they love to have but do not exist currently.
\end{enumerate}
\end{frame}

\begin{frame}
\frametitle{Resources for further study and practice}
\begin{itemize}
\item Dive Into Python3 - free ebook \\ (\url{http://www.diveintopython3.net/})
\item Programming for Linguists, Stanford class (old class, but useful material) \\ \url{http://web.stanford.edu/class/linguist278/}
\item Coursera courses on Python by Charles Severence
\item Other online courses on Python and Intro to CS kind of courses
\end{itemize}
\end{frame}

\begin{frame}
\frametitle{Finally, }
next year this time - I hope to see 2 or 3 of you presenting at a conference something you did only because you know how to program (..and giving a presentation in next year's 516)
\end{frame}

\begin{frame}
\frametitle{Good luck!}
\begin{itemize}
\item Have fun with Python and good luck for your future!
\item My email: sowmya@iastate.edu, vbsowmya@gmail.com 
\item If you need a reference in future, you can contact me (there is no guarantee I will write a letter for each person who asks me, though!)
\end{itemize}
\end{frame}

\begin{frame}
\frametitle{}
\centering
\Large Thank you for your attention! \\
\bigskip 
(may be this gave you new ideas about research/life)  \\
\bigskip
(.. and may be you will thank me after few years!!)
\end{frame}

\end{document}
