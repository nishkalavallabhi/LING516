\documentclass{beamer}
\usepackage[utf8]{inputenc}
\usepackage{graphicx}
\author[Sowmya Vajjala]{Instructor: Sowmya Vajjala}

\title[ENGL 516]{ENGL 516: Methods of Formal Linguistic Analysis}
\subtitle{Semester: Spring '18}

\date{10 Apr 2018}

\institute{Iowa State University, USA}
%%%%%%%%%%%%%%%%%%%%%%%%%%%

\begin{document}

\begin{frame}\titlepage
\end{frame}

\begin{frame}
\frametitle{Class Outline}
\begin{itemize}
\item Assignment 6 Discussion
\item Assignment 7 Description
\item Discussion about your final projects
\item Practice exercise \end{itemize}
\end{frame}

\begin{frame}
\frametitle{}
\Large Assignment 6 discussion
\\ \small (Volunteer team needed)
\end{frame}

\begin{frame}
\frametitle{Assignment 7 description}
\begin{itemize}
\item 10\% of your grade - 2 questions
\item Related to using non-default Python modules (BeautifulSoup and language tool related libraries)
\item If you don't want to use them, you can use whatever you want and write Python code to answer the same questions.
\item For language tool related stuff: if you see errors in installation, write about what did not work, what you did to try to overcome that etc. 
\end{itemize}
\end{frame}

\begin{frame}
\frametitle{Final projects status discussion}
\begin{itemize}
\item What are you doing?
\item How much is already done?
\item What challenges are you facing?
\end{itemize}
\end{frame}

\begin{frame}
\frametitle{Concepts revision via classroom exercise}
I want a program, that takes as input a .txt file path from the user, and prints the following: 
\begin{itemize}
\item Number of unique words/total number of words
\item Percentage of words in this document that appeared only once.
\item Average word frequency for top 25 words.
\end{itemize}
(depending on time, we will see how much we can do today)
\end{frame}

\begin{frame}
\frametitle{Step 1: Think, make notes}
\begin{enumerate}
\item You can discuss among yourself (groups of 2-3) how to write this program.
\item After 15 minutes, I will randomly pick one person for each sub-question - you have to explain to us how you will write your code
\item You can use whiteboard, or plug in your laptop and do live coding if you are sure.
\end{enumerate}
\end{frame}

\begin{frame}
\frametitle{Step 2: Now, write the program}
\begin{enumerate}
\item You can reuse your code from previous classes/assignments (e.g., Assignment 4)
\item You can use unix commands (like in Assignment 5) instead of writing a python program, if you want.
\item If you finish this, you can upload your solutions to the forum with today's date.
\end{enumerate}
(We will continue with this on Thursday)
\end{frame}

\begin{frame}
\frametitle{Next Class}
\begin{itemize}
\item Continuation of today's classroom project
\item Some practice with user interfaces (keeping final projects in mind)
\item Post in "What concepts do you want me to revise?" forum for additional topics
\end{itemize}
\end{frame}

\end{document}
