\documentclass{beamer}
\usepackage[utf8]{inputenc}

\author[Sowmya Vajjala]{Instructor: Sowmya Vajjala}


\title[ENGL 516X]{ENGL 516X: \\ Methods of Formal Linguistic Analysis}
\subtitle{Semester: Spring '18}

\date{6 March 2018}

\institute{Iowa State University, USA}

%%%%%%%%%%%%%%%%%%%%%%%%%%%

\begin{document}

\begin{frame}\titlepage
\end{frame}

\begin{frame}%2minutes
\frametitle{Class outline}
\begin{itemize}
\item Web applications with Bottle - continuation
\item Small exercises and practice
\item Midterm feedback
\end{itemize}
\end{frame}

\begin{frame}
\frametitle{}
\begin{center}
Web application development with Bottle
\end{center}
\end{frame}

%1-2 review slides
\begin{frame}
\frametitle{Web-applications with Bottle}
\begin{itemize}
\item @route
\item @get, @post 
\item Python functions that implement the logic (same as in regular programs)
\item request - to read input data from browser into Python code
\item html (to render stuff on the browser to the user)
\item run()
\end{itemize}
\end{frame}

\begin{frame}
\frametitle{Exercise-1 from previous class}
\begin{itemize}
\item write a Python program using Bottle, that shows two text fields and a button in the initial screen: First Name, Last Name:
\item In the program, have a function, that gets called when the user submits this information, which takes these two values, and returns - ”Hello” + FirstName + LastName on the screen
\end{itemize}
\end{frame}
%Discussion about tuesday's question

\begin{frame}[fragile]
\frametitle{Emily's Solution}
\scriptsize
\begin{verbatim}
from bottle import get, post, request, route, run

@route("/")
@get("/name") # or @route(’/name’)
def get_name():
    return '''
            <form action="/name" method="post">
                First Name: <input name="firstname" type="text" />
                Last Name: <input name="lastname" type="text" />
                <input value="Name" type="submit" />
            </form>
            '''
@post("/name") # or @route(’/name’, method=’POST’)
def hello_user():
    firstname = request.forms.get("firstname")
    lastname = request.forms.get("lastname")
    return "Hello", " ", firstname, " ", lastname

run()
\end{verbatim}
\end{frame}

\begin{frame}
\frametitle{Exercise-2 from previous class}
\begin{itemize}
\item Write a Python program using Bottle, that shows two textfields and a button in the initial screen to take two strings as input
\item In the program, have a function, that gets called when the user submits this information, which takes these two values, and does the following:
\begin{enumerate}
\item checks if the two words are permutations of each other (i.e.,god-dog; spam-maps, program-magropr etc.)
\item Shows a message:  ”The words are permutations” or ”The words are not permutations” depending on the result of the check.
\item Note:  Inputs don’t have to be valid words.  Any word stringsare okay, and you can ignore punctuation, spaces, numbers etc.
\end{enumerate}\end{itemize}
\end{frame}

\begin{frame}[fragile]
\frametitle{Tim's Solution}
\tiny
\begin{verbatim}
from bottle import get, post, request, route, template, run

def permutation(first, second):
    if len(first) != len(second):
        return False
    else:
        return ' '.join(sorted(first)) == ' '.join(sorted(second))

@route('/')
@route('/hello')
def hello():
    name = 'Guest'
    return template('Hello {{name}}', name=name)

@get('/login') # or @route('/login')
def login():
    return '''
        <form action="/login" method="post">
            First string: <input name="first" type="text" />
            Second string: <input name="second" type="text" />
            <input value="Submit" type="submit" />
        </form>
    '''

@post('/login') # or @route('/login', method='POST')
def do_login():
    first = request.forms.get('first')
    second = request.forms.get('second')
    if permutation(first, second) == True:
        return "The strings are permutations."
    else:
        return "The strings are not permutations."

run()
\end{verbatim}
\end{frame}
%.tpl files example

\begin{frame}[fragile]
\frametitle{"Template" files in bottle}
\begin{itemize}
\item files with .tpl extension (you can type in pycharm or notepad and save as somename.tpl)
\item Why? Avoiding typing of all html in the program itself, and storing it separately. 
\item Good thing: While most of the html is static, we can actually modify it based on program output
\item Let me modify Tim's solution and show this.
\end{itemize}
\end{frame}

\begin{frame}[fragile]
\scriptsize
\begin{verbatim}
from bottle import get, post, request, route, template, run

def permutation(first, second):
    if len(first) != len(second):
        return False
    else:
        return ' '.join(sorted(first)) == ' '.join(sorted(second))

@get('/')
def enter():
    return template('homepage.tpl')

@post('/check')
def checkpermutations():
    first = request.forms.get('first')
    second = request.forms.get('second')
    areornot = ""
    if permutation(first, second) == True:
	areornot = "are"
    else:
	areornot = "arenot"
    return template('resultpage.tpl', string1=first, string2=second, areornot=areornot)

run()
\end{verbatim}
\end{frame}

\begin{frame}
\frametitle{}
\begin{center}
Few exercises and discussion
\end{center}
\end{frame}

\begin{frame}[fragile]
\frametitle{What will the following things print?}
\begin{itemize}
\item 
\begin{verbatim}
lst1 = ["this", "is", "a", "list"]
print(lst1[10])
print(lst1[10:])
\end{verbatim} \pause
\item 
\begin{verbatim}
lst1 = [1,"a","b","cc",11]
lst1.sort()
print(lst1)
\end{verbatim} \pause
\item Will this work?
\begin{verbatim}
a = 1
b = 2
a, b = a+1, b+1
print(a)
\end{verbatim}
\end{itemize}
\end{frame}

\begin{frame}[fragile]
\frametitle{Errors and types}
\begin{itemize}
\item What errors do you see here?
\begin{verbatim}
dict1 = {"a":1, "b":2, "c":3}
print(dict1[a])
print(dict1[3])
\end{verbatim} \pause
\item What is a "Syntax Error"?
\item What is a "Type Error"?
\item What is a "Zero Division Error"?
\item What is a "Index Error"?
\end{itemize}
Error Hierarchy in Python: \url{https://docs.python.org/3/library/exceptions.html}
\end{frame}

\begin{frame}[fragile]
\frametitle{What is happening with this loop?}
\begin{verbatim}
list1 = [1,2,3,4,5,6,7]
for item in list1:
  del[list1[item]]
  print(list1)
\end{verbatim}
How many times will it run and what will be printed?
\end{frame}

\begin{frame}
\frametitle{RegEx exercise: pluralize nouns}
Write a small program now using regex, which takes a word as input and returns the plural word as output. Here are the rules to code:
\begin{itemize}
\item If a word ends in s, x, or z, add es to the end of the word.
\item If a word ends in a consonant +y, add ies to the word. If a word ends in a vowel+y, add s in the end (vacancy is vacancies but day is days) 
\item If none of the above cases are valid for a word, just add s at the end of the word and be happy with that. 
\item Post your solution on the forum for today
\end{itemize}
(Your input should not be a number or a string with numbers, punctuation etc. It has to be an alphabetic string.)
\end{frame}

\begin{frame}
\frametitle{}
\begin{center}
Mid-term feedback - please fill it up
\end{center}
\end{frame}

\begin{frame}
\frametitle{Next Class}
\begin{itemize}
\item Topic: Final projects discussion
\item Todo for you: Think about final projects and get back with your ideas - I will ask all groups to talk about their ideas.
\item Look at the descriptions uploaded on Canvas
\item Assignment 5 description
\end{itemize}
\end{frame}

\end{document}
																																																																													
