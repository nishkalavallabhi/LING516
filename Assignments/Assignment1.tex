\documentclass[11pt,a4paper]{article}

\begin{document}
\begin{center}
  Spring Semester 2018 \\ Iowa State University\\[3ex]
  {\large ENGL 516 - Formal Methods of Linguistic Analysis}\\[3ex]
  \textbf{Assignment 1} \\ \textbf{Submission Deadline: 20 Jan 2018}
\end{center}


\paragraph{Instructions:} This assignment consists of five questions. Each question carries 1\% of your final grade. Three of these questions require you to use Python console. Submit your answers in a single PDF file, by adding answers in the format: A1...A5. Please do not paste questions. 

\paragraph{Questions:}

\begin{enumerate}
\item how do you print: 
"I am a graduate student in ISU. \newline
I am learning to program in Python" 
in Python with one print statement? The output has to come in two lines like shown (without quotes), but you should only use one print statement.

\item Declare the following variables in your Python console: \begin{verbatim}
String1 = "Iowa"
String2 = "Ames"
Num1 = 2
Num2 = 4
\end{verbatim}
What happens when you type the following in console:
\begin{verbatim}
String1 + String2
String1* String2
String1+Num1
String1*Num2
\end{verbatim}
If you got an error while executing any of the following statements, what is the error? Why do you think you got it?

\item What do the following statements output in Python console? 
\begin{verbatim}
max(1,2)
max("Language","Linguistics")
min("Language","language")
max("Language",1)
\end{verbatim}
-Which of these threw an error? What exactly are these statements doing, from what you know so far? 
(type these statements instead of copy-pasting - the quotes don't paste properly sometimes)

\item Let us say I am developing a small web-application which collects data from students. I ask students two questions - 
a) Enter a number, b) Enter another number. Once the students enter two numbers, I will show them the sum of these two numbers. Now, you all know how students are.
They don't always read instructions properly, and even if they read, there are those mischevious people. Write an algorithm or draw a flowchart to explain how do you make sure you ensure they entered only numbers, not names or something.

\item Let us say we live in a world where all plurals in English are formed by adding an -es at the end of a word. Write an algorithm or draw a flow-chart for a possible program that takes a word as input and shows its plural as output.

\end{enumerate}


\end{document}
