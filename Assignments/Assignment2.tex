\documentclass[11pt,a4paper]{article}

\begin{document}
\begin{center}
  Spring Semester 2018 \\ Iowa State University\\[3ex]
  {\large ENGL 516 - Formal Methods of Linguistic Analysis}\\[3ex]
  \textbf{Assignment 2} \\ \textbf{Submission Deadline: 3 Feb 2018}
\end{center}


\paragraph{Instructions:} This assignment consists of 4 questions. The first three questions carry 2\% of your final grade and the last one carries 4\%. The questions require you to write Python code or type some commands on Python console. For questions that ask you to write a python code, save files as questionNum.py. For other questions, save all answers as one pdf. Upload all these files separately i.e., do not zip them. If the program file does not run and throws errors, you cannot get a grade unless you defend it in the office hours. I am providing a few test cases for each question, as a means to check if your code works.

\begin{enumerate}
\item Write the python program version of Q4 in Assignment 1. I will provide you a starter snippet, which asks for two numeric inputs from user. Your task is to edit what I gave you, and make it work.

\item Write a program that accomplishes Q5 in Assignment 1. No starter code is provided, you just need to take a string from user as input, append "es" to that string, and print it as output. You have to make sure these are actual strings, and does not contain any numbers or symbols. You can use isalpha() function for strings in the same way as in a2starter code to find this. 
Some test cases are: "car" should become "cares". but "ca1" should print a message "please enter a proper string"

\item Write a program that takes a number, and prints out whether it is even number or odd number. 

\item Find answers to the following using Python console and note the output:
\begin{enumerate}
\item What does this expression print as output on Python console? \textbf{6 * 1 - 2}. What change should you make to make it print -6?

\item On python console, what will you type to: a) find out whether "PYTHON" is upper cased   b) Lowercase the string, and make it "python"  

\item Print the square and cube values of 6.

\item Number of characters in the string: "Python"
\end{enumerate}
\end{enumerate}




\end{document}

\section*{Question 5} 
Write a Python program that takes as input a whole number (say X). Now, write functions for the following in your code:
\begin{enumerate}
  \item Check whether X is a prime number.
  \item Sum of all numbers up to X
  \item Print the square of the number (X*X)
  \item Print the cube of a number (X*X*X)
  \item Print the sum of all digits in the number
\end{enumerate}
Your program should take the input and print the output of all the 5 functions one after another. Include exception handling in your code. Here is how the program output should look like. Square brackets indicate values we fill in while running the program.
